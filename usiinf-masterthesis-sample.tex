\documentclass[mscthesis]{usiinfthesis}
\usepackage{lipsum}
\usepackage{float}
\usepackage{seqsplit}

\usepackage{listings}
\newtheorem{definition}{Definition}[chapter]
\lstdefinelanguage{algebra}
{morekeywords={import,sort,constructors,observers,transformers,axioms,if,
else,end},
sensitive=false,
morecomment=[l]{//s},
}



\title{Solution to contact problem for non-linear thermo-viscoelastic materials in liquid diaphragm pumps} %compulsory
\specialization{Computational Science}%optional
\author{Haoyu Yang} %compulsory
\begin{committee}
\advisor{Prof.}{Krause}{Rolf} %compulsory
\coadvisor{Dr.}{Zulian}{Patrick}{} %optional
\end{committee}
% \Day{September}{ %compulsory
\Month{September} %compulsory
\Year{2025} %compulsory, put only the year
\place{Lugano} %compulsory

\dedication{To my beloved} %optional
\openepigraph{Someone said \dots}{Someone} %optional

%\makeindex %optional, also comment out \theindex at the end

\begin{document}

\maketitle %generates the titlepage, this is FIXED

\frontmatter %generates the frontmatter, this is FIXED

\begin{abstract}
This is a very abstract abstract. 

\end{abstract}

% \begin{abstract}[Zusammenfassung]
% optional, use only if your external advisor requires it in his/er
% language 
% \\

% \lipsum
% \end{abstract}

\begin{acknowledgements}

\end{acknowledgements}

\tableofcontents 
\listoffigures %optional
\listoftables %optional

\mainmatter

% ============================
% Introduction (LaTeX)
% Works with usiinfthesis + natbib (already loaded by the class)
% ============================

\chapter{Introduction}
\label{ch:introduction}

Elastomeric components such as rubber diaphragms and valves are key to the operation of
reciprocating diaphragm pumps \cite{Swienty2024ICEC}. These parts undergo large, reversible
deformations as the pump cycles, and they serve critical functions like sealing to prevent leakage.
The ability of elastomers to sustain such extreme strains while repeatedly flexing makes them ideal
for pump diaphragms, but it also poses significant challenges for simulation and design. A diaphragm
experiences nonlinear elasticity at large strains and pronounced rate-dependent viscoelastic effects,
meaning its stress response depends on the speed and history of deformation. In high-frequency pump
applications, this leads to energy dissipation and stress relaxation that must be captured
by the material model. At the same time, simulating an entire pump assembly with fine finite element
meshes and many loading cycles is computationally intensive, straining both the memory and
processing capabilities of standard solvers. The result is a strong industrial motivation to develop
advanced constitutive models and numerical methods that can predict elastomer behavior in pumps
with high fidelity and efficiency.\\

\noindent Improving these simulations has direct societal and economic relevance: it enables virtual prototyping
of pumps  and helps engineers
optimize elastomer designs for durability and sealing performance without exhaustive physical testing
\cite{Swienty2024ICEC}. In summary, the need to accurately represent complex polymer physics under real operating conditions, combined with the
need for robust and fast computation, defines the central challenge addressed by this work.\\

\noindent A number of physical phenomena govern the mechanics of elastomers in such applications.
Hyperelasticity dominates the instantaneously recoverable response -- even soft rubber can support
high stresses at finite strains by storing elastic energy. Viscoelasticity, on the other hand, introduces
time-dependent behavior: under sustained deformation, stress relaxes, and under cyclic loads, energy
is dissipated as heat. Together these lead to the characteristic hysteresis and stress relaxation observed
in pump diaphragms made of materials like EPDM or polyurethane \cite{Somarathna2020}.
Furthermore, elastomer behavior is highly rate-sensitive. At low strain rates they exhibit rubber-like
compliance, whereas at higher rates they stiffen and can even appear leather-like. Somarathna et al.
(2020) compiled several hyper-viscoelastic models for polyurethane and showed that capturing such
rate-dependent transitions is essential for accurate predictions across different loading speeds
\cite{Somarathna2020}.\\

\noindent Another crucial aspect is the temperature dependence of viscoelastic response. Pump diaphragms may
heat up from internal damping, or experience ambient temperature swings, which shifts their stiffness
and damping characteristics. A common engineering approach to account for this is the
time--temperature superposition principle (TTSP), wherein viscoelastic moduli at various temperatures
are superposed by shifting along the time axis. This effectively defines a shift factor as a function of
temperature, allowing a master relaxation curve to represent a family of curves. TTSP is widely used for
polymers and is the basis for defining temperature-dependent material parameters in this thesis.
However, it comes with limitations: assumptions of thermorheological simplicity often break down for
filled elastomers or thermoplastics, leading to errors outside a limited temperature--frequency range.
Indeed, recent studies have highlighted that while TTSP can extend viscoelastic data, its applicability
must be carefully validated for the specific polymer and deformation mode in question
\cite{Dorleans2021,Ionita2020}. In the present work, temperature effects are incorporated in a weakly
coupled manner -- i.e.\ through temperature-dependent material properties
rather than a fully coupled thermomechanical analysis. This provides a practical compromise: the
material model can reflect the stiffening of the viscoelastic response at lower temperatures or its
softening at higher temperatures \cite{Dorleans2021,Ionita2020}, without the considerable complexity
of solving the heat equation alongside mechanics. The focus remains on capturing the key phenomena
of finite-strain elasticity, rate-dependent viscoelastic dissipation, and temperature-modulated behavior,
as all are vital to predicting how an elastomeric pump component will perform in service.\\

\noindent Parallel to these physical considerations, there is a rich literature on constitutive modeling of finite-strain viscoelasticity that establishes the theoretical foundation for this work. Early pioneering
frameworks by Simo (1987) and Reese \& Govindjee (1998) introduced thermodynamically consistent
models for large-deformation viscoelastic behavior \cite{Simo1987,ReeseGovindjee1998}. These models
are built within the continuum mechanics of irreversible processes: a free energy function is augmented with internal state variables that represent the viscous mechanisms, and a dissipation
potential or evolution law is defined for those internal variables. This internal-variable approach, also
adopted by Holzapfel (1996) for elastomeric structures, has crucial advantages \cite{Holzapfel1996}.
First, it ensures compliance with the second law of thermodynamics -- the formulated models
inherently produce non-negative dissipation, a property essential for realistic
long-term behavior. Second, it provides a tractable way to represent hereditary effects  without storing the entire strain history. Instead, a finite set of internal variables characterizes the current viscoelastic state.\\

\noindent As described in a modern overview by Hackett (2016), the internal variables are approximated by a
recursive expression, and solving the hereditary integrals in incremental form becomes feasible
\cite{Hackett2016}. In practical terms, this means that the convolution integral of past stresses is transformed into update formulas for internal stress-like variables -- a numerical
scheme often known as the Prony series or generalized Maxwell recursion. Each internal variable
typically decays exponentially, and the next increment's
value can be updated from the previous step's value without integrating over the full past. This
approach drastically reduces the computational cost of incorporating viscoelastic memory and has
become standard in finite-strain viscoelastic FEM implementations
\cite{Simo1987,ReeseGovindjee1998,Holzapfel1996,Hackett2016}.\\

\noindent Reese and Govindjee's model in particular utilized a nonlinear evolution law for the internal variables,
enabling it to capture far-from-equilibrium material behavior beyond the reach of linear viscoelastic
approximations \cite{ReeseGovindjee1998}. Subsequent works built on this, for example by including
damage in the viscoelastic network as Simo did, or by
introducing flow rules analogous to plasticity for the viscous strains as in some quasi-linear models.
More recently, Jinaga et al.\ (2025) proposed a comprehensive finite-strain thermomechanical
viscoelastic--viscoplastic model for polymeric materials \cite{Jinaga2025}. Their framework, while very
elaborate, underscores the state of the art in constitutive modeling: it incorporates temperature
dependence and even a viscoplastic flow element on top of nonlinear
viscoelasticity. This thesis is conceptually aligned with such frameworks -- especially in using internal
variable evolution equations for viscoelasticity -- but stays within the viscoelastic regime
and a moderate level of complexity.\\

\noindent The chosen model can be seen as a generalized Maxwell chain at finite strain, calibrated to match the
hyperelastic long-term response and a discrete relaxation spectrum of the material. The relaxation
spectrum is represented by a Prony series whose parameters are obtained by fitting experimental data -- a process for which standard methods exist, including iterative algorithms
to determine Prony coefficients from time-varying strain data \cite{Chen2000NASA}. Notably, an
alternative approach is to infer a continuous relaxation spectrum and then discretize it optimally; this
was demonstrated by Esposito et al.\ (2020) for polymeric foams \cite{Esposito2020}. In our case, we
adopt a more straightforward identification of a finite Prony series, but the model could equivalently
accommodate a spectrum derived from such continuous-to-discrete techniques. In summary, the
literature provides a strong basis -- from thermodynamic consistency and internal-variable formulation
to advanced coupling options -- and our constitutive implementation builds directly upon those
well-established principles.\\

\noindent Implementing a complex finite-strain viscoelastic model in a nonlinear finite element simulation raises
several numerical challenges. One major challenge is the time integration of the internal variable
evolution equations. Because the stiffness of the internal variables' ODEs can vary widely, an implicit integration scheme is usually preferred
for stability. The exponential integrator or recursive update mentioned earlier is one such implicit
scheme that is unconditionally stable for linear viscoelasticity and was extended to the nonlinear case
by Simo and Reese \cite{Simo1987,ReeseGovindjee1998}. Ensuring stability and accuracy of the internal
updates is critical for the overall Newton--Raphson iteration to converge, especially when many internal
variables are active.\\

\noindent Another challenge is computing the consistent tangent stiffness matrix for the Newton
solver. Finite-strain viscoelastic models typically require analytical linearization of both the hyperelastic stress
and the internal variable update to form a tangent moduli tensor consistent with the integration scheme.
If the algorithmic tangent is not consistent, the global Newton iterations may stagnate or diverge,
particularly under large deformation steps. Simo (1987) already emphasized such computational
aspects by deriving the exact tangent for his model, and Reese \& Govindjee (1998) likewise provided
detailed algorithmic consistency in their finite viscoelasticity theory \cite{Simo1987,ReeseGovindjee1998}.
In the present implementation, a fully consistent Jacobian matrix is derived for the coupled hyper-
viscoelastic material routine, which is critical for the quadratic convergence of Newton's method in our
large-strain finite element context.\\

\noindent Beyond correctness and stability, computational efficiency is paramount -- especially given the eventual
goal of simulating full pump systems or large portions thereof. A straightforward finite element
implementation of a hyper-viscoelastic material can be extremely demanding: every integration
point not only stores the usual elastic strain measures but also a set of internal variables for the
viscoelastic state. For a generalized Maxwell model with $N$ Prony terms, each integration point carries
$N$ extra tensor variables. In a three-dimensional mesh with
millions of elements, the memory required to store these history variables can become the bottleneck,
exceeding CPU cache capacities and stressing memory bandwidth. This memory-intensive nature of
history-dependent simulations often makes them bound by memory transfer speeds rather than raw CPU
throughput.\\

\noindent Modern research into high-performance computing (HPC) for finite elements offers strategies to alleviate
this. One strategy is the use of matrix-free methods, wherein the global stiffness matrix is not assembled
or stored explicitly. Instead, element residuals and stiffness actions are computed on-the-fly during the
solver iterations. Martínez-Frutos \& Herrero-Pérez (2015) demonstrated an efficient matrix-free finite
element scheme on GPUs for a fixed-grid elasticity problem, reducing memory use drastically by
avoiding storage of large sparse matrices \cite{MartinezFrutos2015}. The matrix-free approach is
attractive for our problem because it can similarly avoid storing a full material stiffness matrix that would
anyway change every iteration due to viscoelasticity. By recomputing element contributions as needed, we trade extra computations for reduced memory usage --
a good trade-off on modern architectures where FLOPs are abundant but memory bandwidth is limited.\\

\noindent Another HPC strategy is to exploit GPUs for their parallel throughput,
especially for the linear algebra and element calculations. However, achieving high performance on GPUs
requires optimizing memory access patterns and minimizing data transfer between CPU and GPU. Taylor
et al.\ (2008) demonstrated that even viscoelastic internal-variable updates can be integrated efficiently
in GPU-based solvers \cite{taylor2008}.\\

\noindent Our work follows these insights by offloading the linear viscoelastic operations to the GPU, treating them
in a bulk, matrix-free fashion that maximizes the reuse of data on-device. By contrast, the nonlinear
hyperelastic computations (which involve more complex tensor operations and conditional branches)
are kept on the CPU in our current implementation, as they proved more difficult to vectorize and
parallelize on the GPU. This split design leverages each processor's strengths: the GPU excels at the
dense linear combinatorial updates for the internal variables across all integration points, while the CPU
efficiently handles the sophisticated evaluation of the hyperelastic stress and tangent, benefitting from
mature single-core libraries.\\

\noindent Incorporate algebraic multigrid (AMG) preconditioner within the iterative solver for the
linearized system at each Newton step also significantly accelerates convergence for the ill-conditioned
stiffness matrices that can arise in nearly-incompressible elastomers. These HPC-oriented choices -- matrix-free element calculations, GPU-accelerated
viscoelastic updates, and multigrid-precon-\\ditioned solves -- are aimed at overcoming the memory and
processing challenges inherent in large-scale nonlinear viscoelastic simulations \cite{Schussnig2025}.\\

\noindent Developing efficient GPU kernels typically requires careful low-level optimization -- for instance,
coalescing memory accesses for internal-variable arrays and avoiding thread divergence when applying
different Prony terms. Such low-level GPU optimization techniques have been surveyed extensively by
Hijma et al.\ (2023) \cite{Hijma2023}.\\

\noindent While contact mechanics is not the primary focus of this thesis, it is an important consideration for
future work on pump simulations. Contact of viscoelastic bodies has a long history in mechanics; Lee \&
Radok (1960) provided an early solution for a class of linear viscoelastic contact problems \cite{lee1960}.
In diaphragm pumps, contact between elastomeric components and rigid seats is crucial for sealing; in
future work, incorporating robust contact algorithms together with the present hyper-viscoelastic model
will be a key extension \cite{Swienty2024ICEC}.\\

% \noindent In light of the above context and challenges, there is a clear gap in the current practice: no open-source
% solution presently combines a state-of-the-art finite-strain thermo-viscoelastic constitutive model with a
% robust nonlinear solver and HPC optimizations geared towards industrial pump systems. Commercial
% finite element packages do offer user subroutines for custom viscoelasticity, but these are often
% proprietary, and their efficiency on modern parallel hardware is limited. Academic codes, on the other
% hand, have explored high-order or GPU-based FEM but typically with simpler material models.\\

% \noindent As a result, an engineer attempting to simulate a diaphragm pump under realistic cyclic loads might
% face a dilemma: either simplify the material behavior (losing accuracy in predicting damping and stress
% peaks) or simplify the simulation scope (model only a small segment, or a 2D slice, to keep computation
% feasible). The work presented in this thesis aims to bridge that gap by delivering both: a sophisticated
% material model capturing nonlinear viscoelasticity at finite strains with temperature-dependent
% properties, and an implementation optimized for large-scale computation. The open-source nature of the
% code (developed within the SFEM library \cite{SFEM}) further ensures that the outcomes can be freely used,
% tested, and extended by both academia and industry.

\noindent This thesis develops and validates a finite-strain thermo-viscoelastic material model and implements it
in a modern finite element environment with HPC capabilities. The work encompasses:
(i) formulating a nonlinear hyper-viscoelastic constitutive law (with internal variables for a Prony-series
relaxation spectrum) and implementing it in the open-source SFEM code \cite{SFEM},
(ii) deriving an efficient recursive update algorithm for the internal variables, which is used to integrate
the viscoelastic overstress in each element without storing full strain history,
(iii) incorporating temperature-dependent viscoelastic parameters via TTSP-based shift factors,
(iv) offloading the computationally intensive linear viscoelastic updates to the GPU, using a custom
matrix-free kernel that updates all internal variables in parallel,
(v) ensuring that the overall non-linear hyper-viscoelastic solver runs stably on the CPU, and
(vi) performing an extensive experimental validation in collaboration with an industrial partner: the
model is calibrated against multi-rate test data for pump-grade elastomer and then used to simulate a
diaphragm under realistic loading.\\

\noindent The remainder of this thesis is organized as follows. Chapter 2 introduces the theoretical framework of
finite-strain viscoelasticity, detailing the constitutive model's equations, the internal variable formalism,
and the incorporation of time--temperature superposition for temperature-dependent behavior.
Chapter 3 describes the numerical implementation in the SFEM finite element code: the time-integration
of internal variables, the derivation of the consistent tangent, and the matrix-free GPU-accelerated
solution strategy are presented, along with performance optimization details. In Chapter 4, we present
several numerical examples xxxxxxxxx. Chapter 5 then focuses on the experimental validation: we discuss the material testing campaign,
the parameter identification process for the Prony series, and the simulation of a diaphragm under cyclic
pressure loading. Finally, Chapter 6 concludes the thesis by summarizing the findings, highlighting the
contributions, and outlining future work.




\chapter{Methodology}

This chapter presents the theoretical foundations of the finite-strain hyper--viscoelastic solver developed in this thesis, with a clear path from continuum modeling to the discrete operators used in our Total Lagrangian implementation. We begin with a brief recap of linear elasticity (Sec.~2.1) to fix notation and establish stress and strain measures. We then review linear viscoelasticity and the Prony-series representation (Sec.~2.2), which motivates the internal-variable form adopted later for efficient large-scale finite element implementation.\\

\noindent The main part of the chapter is devoted to finite-strain hyper--viscoelastodynamics (Sec.~2.3). We introduce the Total Lagrangian kinematics and stress measures (Sec.~2.3.1), define the equilibrium hyperelastic response using a decoupled Mooney--Rivlin model (Sec.~2.3.2), and extend it to rate-dependent behavior via a generalized Maxwell chain acting on the isochoric response using Prony internal variables (Sec.~2.3.3). To enable robust Newton iterations, we derive the algorithmic stress and its consistent linearization (Sec.~2.3.4). Finally, we formulate the weak form and the corresponding discrete operators needed for assembly (Sec.~2.3.5), and conclude with the time integration strategy based on Newmark-$\beta$ together with a Newton--Raphson solve at each time step (Sec.~2.3.6).


\section{Linear elasticity under infinitesimal strains}
\label{sec:linear_elasticity}

This section briefly recalls small-strain linear elasticity to establish notation and a baseline
mechanical model used later for comparison and for algorithmic components. Although the elastomeric components considered in this thesis
ultimately require finite-strain hyper-viscoelasticity, the linear theory remains useful as a
reference point and as a stepping stone toward more complex constitutive laws \cite{bower2012}.

\subsection{Kinematics and assumptions}
Let $\boldsymbol{u}:\Omega\rightarrow\mathbb{R}^d$ denote the displacement field in a domain $\Omega$.
Under the assumption of infinitesimal strains, the strain tensor is defined by the symmetric part of
the displacement gradient:
\begin{equation}
\boldsymbol{\varepsilon}(\boldsymbol{u})
=
\frac{1}{2}\left(\nabla \boldsymbol{u} + (\nabla \boldsymbol{u})^{T}\right).
\label{eq:infinitesimal_strain}
\end{equation}
Small-strain linear elasticity is characterized by the following standard assumptions \cite{bower2012}:
\begin{itemize}
  \item \textbf{Reversibility.} After removal of external loads, the body returns to its undeformed configuration.
  \item \textbf{Rate- and history-independence.} The stress depends only on the current strain state; it is independent of loading rate and prior history.
  \item \textbf{Linearity (Hooke's law).} The stress--strain relation is linear within the small-deformation regime.
  \item \textbf{Isotropy.} For materials without preferred directions, the constitutive response is invariant under rigid rotations.
\end{itemize}

\subsection{Constitutive law and elasticity tensor}
The constitutive relation of a linear elastic solid reads
\begin{equation}
\boldsymbol{\sigma}
=
\mathbb{C}:\boldsymbol{\varepsilon},
\qquad
\sigma_{ij} = C_{ijkl}\,\varepsilon_{kl},
\label{eq:linear_elastic_constitutive_tensor}
\end{equation}
where $\boldsymbol{\sigma}$ is the Cauchy stress tensor, $\boldsymbol{\varepsilon}$ is the infinitesimal
strain tensor, and $\mathbb{C}$ is the fourth-order elasticity tensor. For an isotropic material,
$\mathbb{C}$ reduces to the Lam\'e form
\begin{equation}
C_{ijkl}
=
\lambda\,\delta_{ij}\delta_{kl}
+
\mu\,(\delta_{ik}\delta_{jl}+\delta_{il}\delta_{jk}),
\label{eq:lame_form}
\end{equation}
where $\lambda$ and $\mu$ are the Lam\'e parameters, and $\delta_{ij}$ is the Kronecker delta.
Equivalently, \eqref{eq:linear_elastic_constitutive_tensor} can be written compactly as
\begin{equation}
\boldsymbol{\sigma}
=
\lambda\,\mathrm{tr}(\boldsymbol{\varepsilon})\,\mathbf{I}
+
2\mu\,\boldsymbol{\varepsilon}.
\label{eq:linear_elastic_constitutive}
\end{equation}

\subsection{Material parameters}
The Lam\'e parameters are related to Young's modulus $E$, Poisson's ratio $\nu$, the bulk modulus
$K$, and the shear modulus $\mu$ by
\begin{equation}
\mu=\frac{E}{2(1+\nu)},
\qquad
\lambda=\frac{E\nu}{(1+\nu)(1-2\nu)},
\qquad
K=\lambda+\frac{2}{3}\mu.
\label{eq:lame_relations}
\end{equation}
In nearly incompressible materials (a common situation for elastomers), $\nu\to 1/2$ and thus
$K\gg \mu$. This separation of volumetric and deviatoric stiffness will later motivate the explicit
volumetric--isochoric splitting used in the finite-strain hyperelastic formulation.

\subsection{Strain energy density}
For linear elasticity, the stored energy density is quadratic in strain:
\begin{equation}
W(\boldsymbol{\varepsilon})
=
\frac{\lambda}{2}\big(\mathrm{tr}\boldsymbol{\varepsilon}\big)^2
+
\mu\,\boldsymbol{\varepsilon}:\boldsymbol{\varepsilon},
\label{eq:linear_elastic_energy}
\end{equation}
and the constitutive law \eqref{eq:linear_elastic_constitutive} follows from
$\boldsymbol{\sigma}=\partial W/\partial \boldsymbol{\varepsilon}$.
This energy-based viewpoint provides a direct conceptual bridge to hyperelasticity, where the energy
is defined in terms of finite-strain measures and stresses are obtained via appropriate derivatives.\\

\noindent
Linear elasticity provides a compact baseline operator $\mathbb{C}$ and a clear
volumetric/deviatoric structure. In the next section, we revisit viscoelasticity and introduce internal
variables and Prony-series recursion, which will be reused in the finite-strain hyper-viscoelastic
model developed later.

\section{Linear viscoelasticity}
\label{sec:linear_viscoelasticity}

Elastomers and many polymer-based materials exhibit a mechanical response that is neither purely
elastic nor purely viscous. Under loading, part of the deformation is stored and recovered,
while another part is associated with time-dependent molecular rearrangements (viscosity), leading to
\emph{creep} and \emph{stress relaxation}. From an engineering viewpoint, these effects manifest as
rate dependence, hysteresis, and strong temperature
sensitivity—features that are central for predicting the dynamic behaviour of pump-grade elastomeric
components under repeated excitation \cite{bower2012,Haupt2002}.

\subsection{Key phenomena and temperature sensitivity}
Two canonical experiments summarize linear viscoelastic behaviour:
(i) \textbf{creep}: the strain increases in time under constant stress;
(ii) \textbf{stress relaxation}: the stress decreases in time under constant strain.
In dynamic settings, viscoelasticity also implies phase lag between stress and strain, and therefore
cycle-to-cycle energy dissipation. Moreover, polymer stiffness varies markedly with temperature,
especially near transitions such as the glass-transition region. Fig.~\ref{fig:shear_modulus_temp}
illustrates the qualitative dependence of shear modulus on temperature for a typical polymer
\cite{bower2012}. In this thesis, such temperature effects are later incorporated through
\emph{temperature-dependent material parameters}, rather than by solving a coupled
thermal PDE.

\begin{figure}[H]
  \centering
  \includegraphics[width=0.75\linewidth]{Variation_of_shear_modulus_with_diff_T.png}
  \caption{Qualitative variation of shear modulus of a typical polymer with temperature \cite{bower2012}.}
  \label{fig:shear_modulus_temp}
\end{figure}

\subsection{Spring--dashpot representations (mechanical analogs)}
A widely used entry point to linear viscoelasticity is the spring--dashpot analogy, where a linear
spring represents elasticity and a dashpot represents viscosity. Classical two-parameter models include
the Kelvin--Voigt model and the Maxwell mode. By combining multiple Maxwell branches in parallel, one obtains the \emph{generalized Maxwell}
model, which naturally leads to a Prony-series representation of relaxation \cite{bower2012,Haupt2002}.

\begin{figure}[H]
  \centering
  \includegraphics[width=0.75\linewidth]{dff_spring_dashpot.png}
  \caption{Spring--dashpot representations commonly used to model polymer viscoelasticity.}
  \label{fig:spring_dashpot_models}
\end{figure}

\subsection{Kelvin--Voigt model}
In the Kelvin--Voigt model, a spring of stiffness $E$ and a dashpot of viscosity $\eta$ are connected in
parallel. The strain is shared by both elements, while stresses add up, yielding
\begin{equation}
  \boldsymbol{\sigma}(t) = E\,\boldsymbol{\varepsilon}(t) + \eta\,\dot{\boldsymbol{\varepsilon}}(t).
  \label{eq:kelvin_voigt}
\end{equation}
This model captures \emph{creep} with a bounded long-time strain under a constant applied stress, but it
does not reproduce an instantaneous stress jump under an imposed strain history and therefore
cannot represent stress relaxation accurately \cite{bower2012}. In practice, Kelvin--Voigt is often used as
a simple damping model, or as a building block when viscoelasticity is embedded into a broader
framework.

\subsection{Maxwell model}
In the Maxwell model, the spring and dashpot are connected in series. The stress is common to both
elements, while total strain is additive, which leads to
\begin{equation}
  \dot{\boldsymbol{\varepsilon}}(t) = \frac{\boldsymbol{\sigma}(t)}{\eta}
  + \frac{1}{E}\dot{\boldsymbol{\sigma}}(t).
  \label{eq:maxwell}
\end{equation}
The Maxwell model reproduces \emph{stress relaxation} naturally: under a suddenly imposed constant
strain, the stress decays exponentially with a characteristic relaxation time $\tau=\eta/E$
\cite{Haupt2002}. However, under constant stress it predicts unbounded creep. This motivates the use
of multiple relaxation mechanisms in parallel.

\subsection{Generalized Maxwell model and Prony-series representation}
Real polymers exhibit a \emph{spectrum} of relaxation times rather than a single relaxation time. A common
linear model is the generalized Maxwell model, consisting of $N$ Maxwell branches in parallel with an
equilibrium spring. Fig.~\ref{fig:generalised_Maxwell_model} illustrates this construction.

\begin{figure}[H]
  \centering
  \includegraphics[width=0.75\linewidth]{generallized_maxwell.png}
  \caption{Generalized Maxwell model (Prony-series representation) \cite{Penghao2024}.}
  \label{fig:generalised_Maxwell_model}
\end{figure}

\noindent
In a compact form, the total stress is decomposed into an equilibrium contribution and viscous
overstresses:
\begin{equation}
  \boldsymbol{\sigma} = E_\infty \boldsymbol{\varepsilon} + \sum_{i=1}^{N} \boldsymbol{q}_i,
  \label{eq:total_stress}
\end{equation}
where $\boldsymbol{q}_i$ denotes the internal (branch) stress of the $i$-th Maxwell element, and
$E_\infty$ is the long-time (equilibrium) modulus. Using the internal strain variable
$\boldsymbol{\alpha}_i$ for the dashpot strain, one may write the branch relation as
\begin{equation}
  \boldsymbol{q}_i(t) = \eta_i\,\dot{\boldsymbol{\alpha}}_i(t) = E_i(\boldsymbol{\varepsilon}-\boldsymbol{\alpha}_i).
  \label{eq:prony_internal}
\end{equation}
Differentiating \eqref{eq:prony_internal} and eliminating $\dot{\boldsymbol{\alpha}}_i$ yields the
standard first-order evolution equation
\begin{equation}
  \dot{\boldsymbol{q}}_i(t) + \frac{\boldsymbol{q}_i}{\tau_i} = \frac{d}{dt}\left(E_i\boldsymbol{\varepsilon}\right),
  \qquad \tau_i = \eta_i/E_i.
  \label{eq:prony_evolution}
\end{equation}
Equation \eqref{eq:prony_evolution} is linear and admits a convolution representation. Using an
integrating-factor argument (and assuming $\boldsymbol{q}_i(0)=\mathbf{0}$ for simplicity) one obtains
\begin{align}
  \boldsymbol{q}_i(t)
  &= E_i \exp(-t/\tau_i)\int_0^t \exp(s/\tau_i)\,\dot{\boldsymbol{\varepsilon}}(s)\,ds
  \label{eq:q_i_initial}
\end{align}
and equivalently the hereditary (Boltzmann-type) form
\begin{equation}
  \boldsymbol{q}_i(t) = E_i \int_0^t \exp\!\left(-\frac{t-s}{\tau_i}\right)\dot{\boldsymbol{\varepsilon}}(s)\,ds.
  \label{eq:q_i_conv_non_{node}orm}
\end{equation}

\subsubsection{Normalized Prony coefficients}
Define the instantaneous modulus
\begin{equation}
  E_0 = E_\infty + \sum_{i=1}^{N} E_i,
\end{equation}
and normalized Prony coefficients
\begin{equation}
  g_\infty=\frac{E_\infty}{E_0},
  \qquad
  g_i=\frac{E_i}{E_0},
  \qquad
  g_\infty+\sum_{i=1}^N g_i = 1.
\end{equation}
Then \eqref{eq:total_stress} can be rewritten as
\begin{equation}
  \boldsymbol{\sigma} = g_\infty E_0 \boldsymbol{\varepsilon} + \sum_{i=1}^{N} \boldsymbol{q}_i,
\end{equation}
where the branch stress takes the normalized convolution form
\begin{equation}
  \boldsymbol{q}_i(t) = g_i E_0 \int_0^t \exp\!\left(-\frac{t-s}{\tau_i}\right)\dot{\boldsymbol{\varepsilon}}(s)\,ds.
  \label{eq:q_i}
\end{equation}

\subsubsection{Time interval partitioning and recursive update}
We now have a closed formula between $\boldsymbol{q}_i$ and $\boldsymbol{\varepsilon}$, but since
\eqref{eq:q_i} is a convolution integral, evaluating it from $0$ to $t$ at every time step is not practical.
A standard remedy is to derive a \emph{recursive} update, which avoids storing the full strain history and
only requires the previous internal state. Following the discretization strategy also discussed in
\cite{Penghao2024}, consider a time interval $[t_n,t_{n+1}]$ with $\Delta t=t_{n+1}-t_n$ and split
\eqref{eq:q_i} into a history part and a current-step part:
\begin{align}
\boldsymbol{q}_i(t_{n+1}) &= g_i E_0 \int_0^{t_n} \exp\left[-\frac{t_{n+1}-s}{\tau_i}\right] \frac{d\boldsymbol{\varepsilon}(s)}{ds} ds \notag \\
&\quad + g_i E_0 \int_{t_n}^{t_{n+1}} \exp\left[-\frac{t_{n+1}-s}{\tau_i}\right] \frac{d\boldsymbol{\varepsilon}(s)}{ds} ds.
\end{align}
The first integral can be expressed using $\boldsymbol{q}_i(t_n)$:
\begin{align}
&g_i E_0 \int_0^{t_n} \exp\left[-\frac{t_{n+1}-s}{\tau_i}\right] \frac{d\boldsymbol{\varepsilon}(s)}{ds} ds \notag \\
&= g_i E_0 \exp\left[-\frac{\Delta t}{\tau_i}\right] \int_0^{t_n} \exp\left[-\frac{t_n-s}{\tau_i}\right] \frac{d\boldsymbol{\varepsilon}(s)}{ds} ds \notag \\
&= \exp\left[-\frac{\Delta t}{\tau_i}\right]\boldsymbol{q}_i(t_n).
\end{align}
For the second integral, assume a linear variation of strain over the time step:
\begin{equation}
\boldsymbol{\varepsilon}(s) \approx \boldsymbol{\varepsilon}(t_n) + \frac{s-t_n}{\Delta t}\big[\boldsymbol{\varepsilon}(t_{n+1})-\boldsymbol{\varepsilon}(t_n)\big],
\end{equation}
so that the strain rate is approximately constant:
\begin{equation}
\frac{d\boldsymbol{\varepsilon}(s)}{ds} \approx \frac{\boldsymbol{\varepsilon}(t_{n+1})-\boldsymbol{\varepsilon}(t_n)}{\Delta t}.
\end{equation}
Substituting into the integral yields
\begin{align}
&g_i E_0\int_{t_n}^{t_{n+1}} \exp\left[-\frac{t_{n+1}-s}{\tau_i}\right]
\frac{\boldsymbol{\varepsilon}(t_{n+1})-\boldsymbol{\varepsilon}(t_n)}{\Delta t}\,ds \notag\\
&= g_i E_0\frac{\boldsymbol{\varepsilon}(t_{n+1})-\boldsymbol{\varepsilon}(t_n)}{\Delta t}\,
\tau_i\left[1-\exp\left(-\frac{\Delta t}{\tau_i}\right)\right].
\end{align}
Combining both parts leads to the recursion
\begin{align}
\boldsymbol{q}_i(t_{n+1})
&= \exp\left(-\frac{\Delta t}{\tau_i}\right)\boldsymbol{q}_i(t_n)
+ g_i E_0\frac{\boldsymbol{\varepsilon}(t_{n+1})-\boldsymbol{\varepsilon}(t_n)}{\Delta t}\,
\tau_i\left[1-\exp\left(-\frac{\Delta t}{\tau_i}\right)\right].
\end{align}
Define the coefficients
\begin{align}
\alpha_i &= \exp\left(-\frac{\Delta t}{\tau_i}\right),\\
\beta_i  &= g_i E_0\frac{1-\exp(-\Delta t/\tau_i)}{\Delta t/\tau_i},
\end{align}
and obtain the compact discrete update
\begin{equation}
\boxed{
\begin{cases}
\boldsymbol{q}_i(t_{n+1}) = \alpha_i\,\boldsymbol{q}_i(t_n)
+ \beta_i\,[\boldsymbol{\varepsilon}(t_{n+1})-\boldsymbol{\varepsilon}(t_n)],\\
\alpha_i = \exp(-\Delta t/\tau_i),\\
\beta_i = g_i E_0 \dfrac{1-\exp(-\Delta t/\tau_i)}{\Delta t/\tau_i}.
\end{cases}}
\label{eq:prony_recursion}
\end{equation}

\noindent
The recursion \eqref{eq:prony_recursion} is essential for efficient simulation: it replaces full-history
convolution by a local update per time step, requiring only the previous internal state
$\boldsymbol{q}_i(t_n)$. This formulation will later be reused when constructing a time-discrete,
dynamics-oriented solver where viscoelastic internal variables must be updated consistently at each step.\\

\noindent
The remainder of this chapter moves from constitutive modelling to discretization. We next introduce
the finite element setting and the time-integration strategy used for dynamics (Newmark-$\beta$),
which provides the backbone for transient simulations with viscoelastic constitutive updates.


% ==========================================================
% 2.3 Finite-strain hyper-viscoelastodynamics (mainline)
% 2.3.1 Finite-strain kinematics & stress measures (TL)
% ==========================================================

\section{Finite-strain hyper-viscoelastodynamics}
\label{sec:finite_strain_mainline}

\subsection{Finite-strain kinematics and stress measures (Total Lagrangian)}
\label{sec:finite_strain_kinematics}

\paragraph{Notation and conventions.}
Scalars are denoted by italic letters (e.g.\ $t$), vectors by bold lowercase (e.g.\ $\mathbf{u}$),
and second-order tensors by bold uppercase (e.g.\ $\mathbf{F}$).
The identity tensor is $\mathbf{I}$, the transpose is $(\cdot)^T$,
the trace is $\mathrm{tr}(\cdot)$, and the determinant is $\det(\cdot)$.
For two second-order tensors $\mathbf{A},\mathbf{B}$, the double contraction is
$\mathbf{A}:\mathbf{B} := \sum_{i,j} A_{ij}B_{ij}$.
All gradients and divergences in this section are taken with respect to the \emph{reference}
(material) coordinates $\mathbf{X}$; we denote
\[
\nabla_{\!X}(\cdot) := \frac{\partial(\cdot)}{\partial\mathbf{X}},
\qquad
\mathrm{Div}(\cdot) := \nabla_{\!X}\!\cdot(\cdot).
\]

\paragraph{Reference configuration and motion.}
Let $\Omega_0 \subset \mathbb{R}^3$ be the reference configuration of the body,
with boundary $\partial\Omega_0 = \Gamma_0^{u} \cup \Gamma_0^{t}$, where
$\Gamma_0^{u}$ is the displacement (Dirichlet) boundary and $\Gamma_0^{t}$ is the traction (Neumann) boundary.
Material points are labeled by $\mathbf{X}\in\Omega_0$ and time by $t\in[0,T]$, where $T>0$ is the final time.
The motion map is
\begin{equation}
  \boldsymbol{\varphi}(\mathbf{X},t) : \Omega_0 \times [0,T] \to \mathbb{R}^3,
  \qquad
  \mathbf{x} = \boldsymbol{\varphi}(\mathbf{X},t),
\end{equation}
and the displacement field is defined by
\begin{equation}
  \mathbf{u}(\mathbf{X},t) := \boldsymbol{\varphi}(\mathbf{X},t) - \mathbf{X}.
\end{equation}
In a \emph{Total Lagrangian} formulation, unknowns and test functions are defined on $\Omega_0$,
and all volume integrals are evaluated over $\Omega_0$.

\paragraph{Deformation gradient and Jacobian.}
The deformation gradient is the reference gradient of the motion:
\begin{equation}
  \mathbf{F}(\mathbf{X},t) := \nabla_{\!X}\boldsymbol{\varphi}(\mathbf{X},t)
  = \mathbf{I} + \nabla_{\!X}\mathbf{u}(\mathbf{X},t).
  \label{eq:def_F}
\end{equation}
The local volume change is measured by the Jacobian determinant
\begin{equation}
  J(\mathbf{X},t) := \det\mathbf{F}(\mathbf{X},t),
  \qquad J>0,
  \label{eq:def_J}
\end{equation}
and the reference and current volume elements satisfy $dv = J\,dV$,
where $dV$ denotes an infinitesimal volume in $\Omega_0$ and $dv$ its image in the current configuration.

\paragraph{Strain measure and invariants.}
We introduce the right Cauchy--Green tensor
\begin{equation}
  \mathbf{C}(\mathbf{X},t) := \mathbf{F}^T\mathbf{F}.
  \label{eq:def_C}
\end{equation}
For isotropic materials, it is convenient to use the principal invariants of $\mathbf{C}$:
\begin{equation}
  I_1(\mathbf{C}) := \mathrm{tr}(\mathbf{C}),\qquad
  I_2(\mathbf{C}) := \tfrac{1}{2}\Big[(\mathrm{tr}\,\mathbf{C})^2 - \mathrm{tr}(\mathbf{C}^2)\Big],\qquad
  I_3(\mathbf{C}) := \det(\mathbf{C}) = J^2.
  \label{eq:C_invariants}
\end{equation}

\paragraph{Isochoric--volumetric split (nearly-incompressible elastomers).}
To separate distortional (volume-preserving) deformation from volumetric deformation,
we define the \emph{isochoric} deformation gradient and right Cauchy--Green tensor by
\begin{equation}
  \bar{\mathbf{F}} := J^{-1/3}\mathbf{F},
  \qquad
  \bar{\mathbf{C}} := \bar{\mathbf{F}}^T\bar{\mathbf{F}} = J^{-2/3}\mathbf{C},
  \label{eq:isochoric_split}
\end{equation}
so that $\det\bar{\mathbf{F}}=1$ and $\det\bar{\mathbf{C}}=1$.
The corresponding isochoric invariants are defined as
\begin{equation}
  \bar{I}_1 := \mathrm{tr}(\bar{\mathbf{C}}) = J^{-2/3} I_1,
  \qquad
  \bar{I}_2 := \tfrac{1}{2}\Big[(\mathrm{tr}\,\bar{\mathbf{C}})^2 - \mathrm{tr}(\bar{\mathbf{C}}^2)\Big]
             = J^{-4/3} I_2.
  \label{eq:isochoric_invariants}
\end{equation}
This split will be used in Section~\ref{sec:mooney_rivlin_flexible} to define the equilibrium
hyperelastic energy $W_\infty(\bar{\mathbf{C}},J)$, and in Section~\ref{sec:finite_strain_prony}
to formulate a deviatoric (isochoric) Prony-series overstress.

\paragraph{Stress measures.}
We use the first Piola--Kirchhoff stress $\mathbf{P}(\mathbf{X},t)$, which maps reference area normals to current tractions.
It is related to the second Piola--Kirchhoff stress $\mathbf{S}(\mathbf{X},t)$ by
\begin{equation}
  \mathbf{P} = \mathbf{F}\mathbf{S}.
  \label{eq:P_S_relation}
\end{equation}
For completeness, the Kirchhoff stress $\boldsymbol{\tau}$ and the Cauchy stress $\boldsymbol{\sigma}$ are
\begin{equation}
  \boldsymbol{\tau} := \mathbf{P}\mathbf{F}^T,
  \qquad
  \boldsymbol{\sigma} := \frac{1}{J}\boldsymbol{\tau}.
  \label{eq:tau_sigma}
\end{equation}

\paragraph{Work conjugacy and internal virtual work (Total Lagrangian).}
Let $\delta\mathbf{u}$ be an admissible virtual displacement (test variation) on $\Omega_0$.
The corresponding virtual deformation-gradient increment is
$\delta\mathbf{F} := \nabla_{\!X}\delta\mathbf{u}$.
In the Total Lagrangian setting, the internal virtual work is
\begin{equation}
  \delta W_{\mathrm{int}}
  = \int_{\Omega_0} \mathbf{P} : \delta\mathbf{F}\, dV
  = \int_{\Omega_0} \mathbf{P} : \nabla_{\!X}\delta\mathbf{u}\, dV.
  \label{eq:virtual_work_P}
\end{equation}
This relation fixes the stress measure used in the finite element residual:
$\mathbf{P}$ is \emph{work-conjugate} to $\mathbf{F}$ in $\Omega_0$.

\paragraph{Hyperelastic equilibrium stress from a stored energy.}
If an equilibrium (rate-independent) stored energy density per unit reference volume,
$W(\mathbf{F})$ (or equivalently $W(\mathbf{C})$), is specified, then the associated stresses are
\begin{equation}
  \mathbf{P} := \frac{\partial W}{\partial \mathbf{F}},
  \qquad
  \mathbf{S} := 2\,\frac{\partial W}{\partial \mathbf{C}},
  \label{eq:def_P_S_from_energy}
\end{equation}
with $\mathbf{P}=\mathbf{F}\mathbf{S}$ consistent with \eqref{eq:P_S_relation}.
The specific form of $W$ used in this thesis (Mooney--Rivlin, flexible variant) is introduced next.


\subsection{Hyperelasticity: Mooney--Rivlin}
\label{sec:mr_flexible}

To construct a finite-strain hyper--viscoelastic model, we first define the
\emph{equilibrium} (long-term) response that remains after all viscoelastic overstresses
have decayed. Throughout this thesis, the subscript ``$\infty$'' refers to this
\emph{relaxed equilibrium branch}. Importantly, only the \emph{isochoric} response is allowed
to relax in the present model (Sec.~\ref{sec:fs_visco_prony}), whereas the volumetric response
is kept purely elastic.

\paragraph{Isochoric--volumetric split.}
We adopt the standard decoupled form
\begin{equation}
  W(\mathbf{C})
  = W_{\mathrm{iso}}(\bar{\mathbf{C}}) + W_{\mathrm{vol}}(J),
  \label{eq:W_split}
\end{equation}
with $J=\det\mathbf{F}>0$ and the isochoric right Cauchy--Green tensor
\begin{equation}
  \bar{\mathbf{C}} := J^{-2/3}\mathbf{C}, \qquad \det\bar{\mathbf{C}}=1,
  \label{eq:Cbar_def}
\end{equation}
where $\mathbf{C}=\mathbf{F}^\mathsf{T}\mathbf{F}$. The split isolates the deviatoric response
from the volumetric response.

\paragraph{Mooney--Rivlin isochoric energy.}
For the deviatoric part we use the Mooney--Rivlin model written in terms of the
\emph{isochoric invariants} of $\bar{\mathbf{C}}$:
\begin{equation}
  \bar I_1 := \mathrm{tr}(\bar{\mathbf{C}}),\qquad
  \bar I_2 := \tfrac12\left[(\mathrm{tr}\,\bar{\mathbf{C}})^2-\mathrm{tr}(\bar{\mathbf{C}}^2)\right].
  \label{eq:Ibar_def}
\end{equation}
The isochoric stored energy is
\begin{equation}
  W_{\mathrm{iso}}(\bar{\mathbf{C}})
  = C_{10}\,(\bar I_1-3) + C_{01}\,(\bar I_2-3),
  \label{eq:W_iso_mr}
\end{equation}
where $C_{10}$ and $C_{01}$ are material parameters.

\paragraph{Volumetric penalty (purely elastic).}
To control compressibility we employ the quadratic penalty
\begin{equation}
  W_{\mathrm{vol}}(J) = \frac{1}{D_1}(J-1)^2,
  \label{eq:W_vol_penalty}
\end{equation}
with parameter $D_1>0$. In the small-strain limit this corresponds to an effective bulk modulus
$K \approx 2/D_1$, hence small $D_1$ enforces near-incompressibility.

% \paragraph{Stress measures and notation used in the sequel.}
% Since the energy is defined per unit reference volume, we work with the second and first
% Piola--Kirchhoff stresses
% \begin{equation}
%   \mathbf{S} := 2\,\frac{\partial W}{\partial \mathbf{C}},
%   \qquad
%   \mathbf{P} := \frac{\partial W}{\partial \mathbf{F}}
%   \quad\text{with}\quad
%   \mathbf{P} = \mathbf{F}\,\mathbf{S}.
%   \label{eq:S_P_from_W_general}
% \end{equation}
% In a Total Lagrangian setting, these are the natural measures entering the weak form.

\paragraph{Instantaneous-reference split.}
We define the \emph{instantaneous-reference} stresses associated with the split \eqref{eq:W_split} as
\begin{equation}
  \mathbf{S}_{\mathrm{iso},0} := 2\,\frac{\partial W_{\mathrm{iso}}(\bar{\mathbf{C}})}{\partial \mathbf{C}},
  \qquad
  \mathbf{S}_{\mathrm{vol}} := 2\,\frac{\partial W_{\mathrm{vol}}(J)}{\partial \mathbf{C}},
  \label{eq:S_iso0_S_vol_def}
\end{equation}
and the corresponding first Piola stresses
\begin{equation}
  \mathbf{P}_{\mathrm{iso},0} := \mathbf{F}\mathbf{S}_{\mathrm{iso},0},
  \qquad
  \mathbf{P}_{\mathrm{vol}} := \mathbf{F}\mathbf{S}_{\mathrm{vol}}.
  \label{eq:P_iso0_P_vol_def}
\end{equation}
Here $\mathbf{S}_{\mathrm{iso},0}$ will serve as the driving stress for the Maxwell branches in
Sec.~\ref{sec:fs_visco_prony}. Note that $\mathbf{S}_{\mathrm{vol}}$ is \emph{purely elastic} in the present thesis;
hence we do \emph{not} attach subscripts ``$0$'' or ``$\infty$'' to volumetric quantities.

\paragraph{Closed-form expressions.}
For the Mooney--Rivlin form \eqref{eq:W_iso_mr}, it is convenient to express $\mathbf{S}_{\mathrm{iso},0}$ via
the derivatives of the isochoric invariants with respect to $\mathbf{C}$. Using
$\bar I_1 = J^{-2/3} I_1$ and $\bar I_2 = J^{-4/3} I_2$ with
$I_1=\mathrm{tr}\,\mathbf{C}$ and $I_2=\tfrac12[(\mathrm{tr}\,\mathbf{C})^2-\mathrm{tr}(\mathbf{C}^2)]$,
one obtains the standard relations
\begin{align}
  \frac{\partial \bar I_1}{\partial \mathbf{C}}
  &= J^{-2/3}\left(\mathbf{I} - \frac{1}{3} I_1\,\mathbf{C}^{-1}\right), \label{eq:dIbar1_dC}\\
  \frac{\partial \bar I_2}{\partial \mathbf{C}}
  &= J^{-4/3}\left( I_1\,\mathbf{I} - \mathbf{C} - \frac{2}{3} I_2\,\mathbf{C}^{-1}\right), \label{eq:dIbar2_dC}
\end{align}
and therefore
\begin{equation}
  \mathbf{S}_{\mathrm{iso},0}
  = 2\left(
  C_{10}\,\frac{\partial \bar I_1}{\partial \mathbf{C}}
  + C_{01}\,\frac{\partial \bar I_2}{\partial \mathbf{C}}
  \right).
  \label{eq:S_iso_mr}
\end{equation}
For the volumetric energy \eqref{eq:W_vol_penalty}, using
$\dfrac{\partial J}{\partial \mathbf{C}}=\tfrac12 J\,\mathbf{C}^{-1}$ yields
\begin{equation}
  \mathbf{S}_{\mathrm{vol}}
  = 2\,\frac{\partial W_{\mathrm{vol}}}{\partial \mathbf{C}}
  = 2\left(\frac{2}{D_1}(J-1)\right)\frac{\partial J}{\partial \mathbf{C}}
  = \frac{2J(J-1)}{D_1}\,\mathbf{C}^{-1}.
  \label{eq:S_vol}
\end{equation}

\paragraph{Equilibrium branch used by the hyper--viscoelastic model.}
In the generalized Maxwell model introduced next (Sec.~\ref{sec:fs_visco_prony}), only the isochoric response
relaxes. Accordingly, the long-term (equilibrium) stresses are defined as
\begin{equation}
  \mathbf{S}_{\infty} := \mathbf{S}_{\mathrm{vol}} + g_{\infty}\,\mathbf{S}_{\mathrm{iso},0},
  \qquad
  \mathbf{P}_{\infty} := \mathbf{F}\mathbf{S}_{\infty}.
  \label{eq:Sinf_Pinf_def}
\end{equation}
These stresses will be augmented by viscoelastic overstress contributions $\{\mathbf Q_m\}$ in
Sec.~\ref{sec:fs_visco_prony}, yielding the total stress used in dynamics.


\subsection{Finite-strain viscoelasticity: generalized Maxwell model with Prony internal variables}
\label{sec:fs_visco_prony}

Building on the hyperelastic ingredients introduced in Sec.~\ref{sec:mr_flexible}, we model rate-dependence
through a generalized Maxwell chain acting on the \emph{isochoric} response only, while the volumetric response
remains purely elastic. This choice is standard for rubber-like polymers: time-dependence is mainly associated
with shear mechanisms and it avoids unphysical long-term volumetric relaxation.

\paragraph{Reduced time and Prony series.}
Consistent with Sec.~2.2, we introduce the reduced time increment
\begin{equation}
  \Delta t_r := \frac{\Delta t}{a_T(T)},
  \label{eq:reduced_time}
\end{equation}
where $a_T(T)$ is a temperature shift factor (e.g.\ WLF/Arrhenius). The normalized relaxation function is
approximated by a Prony series
\begin{equation}
  g(t_r) = g_{\infty} + \sum_{m=1}^{N_p} g_m \exp\!\left(-\frac{t_r}{\tau_m}\right),
  \qquad
  g_{\infty} + \sum_{m=1}^{N_p} g_m = 1,
  \label{eq:prony_relax}
\end{equation}
with Prony weights $g_m$ and relaxation times $\tau_m$ (defined at a reference temperature).

\paragraph{Driving stresses in the reference configuration.}
All constitutive quantities are expressed in $\Omega_0$ (Total Lagrangian). We define
\begin{equation}
  \mathbf S_{\mathrm{vol}} := 2\,\frac{\partial W_{\mathrm{vol}}(J)}{\partial \mathbf C},
  \qquad
  \mathbf S_{\mathrm{iso},0} := 2\,\frac{\partial W_{\mathrm{iso}}(\bar{\mathbf C})}{\partial \mathbf C}.
  \label{eq:S_vol_S_iso0}
\end{equation}
Here $\mathbf S_{\mathrm{iso},0}$ is the \emph{instantaneous-reference} isochoric (deviatoric) second
Piola--Kirchhoff stress computed from the same isochoric energy $W_{\mathrm{iso}}$ as in Sec.~\ref{sec:mr_flexible}.
Importantly, the volumetric stress $\mathbf S_{\mathrm{vol}}$ is \emph{purely elastic} in this thesis; therefore we do
not attach subscripts ``$0$'' or ``$\infty$'' to volumetric quantities.

\paragraph{Total stress with overstress internal variables.}
We introduce stress-like internal variables $\{\mathbf Q_m\}_{m=1}^{N_p}$, stored at each quadrature
point in $\Omega_0$. The \emph{total} second Piola--Kirchhoff stress at $t_{n+1}$ is defined as
\begin{equation}
  \mathbf S^{n+1}
  = \mathbf S_{\mathrm{vol}}^{n+1}
  + g_{\infty}\,\mathbf S_{\mathrm{iso},0}^{n+1}
  + \sum_{m=1}^{N_p} \mathbf Q_m^{n+1}.
  \label{eq:S_total_prony}
\end{equation}
The corresponding \emph{total} first Piola--Kirchhoff stress follows from $\mathbf P=\mathbf F\mathbf S$:
\begin{equation}
  \mathbf P^{n+1} = \mathbf F^{n+1}\mathbf S^{n+1}.
  \label{eq:P_total_prony_compact}
\end{equation}
When all overstresses decay ($\mathbf Q_m\!\to\!\mathbf 0$), the long-term response is
$\mathbf S_{\infty}=\mathbf S_{\mathrm{vol}}+g_{\infty}\mathbf S_{\mathrm{iso},0}$ and
$\mathbf P_{\infty}=\mathbf F\mathbf S_{\infty}$.

\paragraph{Evolution equation and exact time-discrete recursion.}
Each Maxwell branch is driven by the rate of the instantaneous-reference isochoric stress:
\begin{equation}
  \dot{\mathbf Q}_m + \frac{1}{\tau_m}\mathbf Q_m = g_m\,\dot{\mathbf S}_{\mathrm{iso},0}.
  \label{eq:Q_ode}
\end{equation}
Assuming $\dot{\mathbf S}_{\mathrm{iso},0}$ is constant within one step $[t_n,t_{n+1}]$, the exact update of
\eqref{eq:Q_ode} yields
\begin{equation}
  \mathbf Q_m^{n+1}
  = \alpha_m\,\mathbf Q_m^{n}
  + \beta_m\left(\mathbf S_{\mathrm{iso},0}^{n+1} - \mathbf S_{\mathrm{iso},0}^{n}\right),
  \label{eq:Q_update}
\end{equation}
with
\begin{equation}
  \alpha_m := \exp\!\left(-\frac{\Delta t_r}{\tau_m}\right),
  \qquad
  \beta_m := g_m\,\frac{\tau_m}{\Delta t_r}\left(1-\alpha_m\right).
  \label{eq:alpha_beta}
\end{equation}
This internal-variable form is equivalent to the convolution representation, while being more suitable
for large-scale implementations because it requires only $N_p$ tensor-valued state variables per quadrature point.



\subsection{Algorithmic stress and consistent linearization}
\label{sec:alg_stress_consistent_lin}

For the Newton solve at time $t_{n+1}$, we require a constitutive update that returns the \emph{algorithmic stress}
$\mathbf P^{n+1}$ and its \emph{consistent} linearization with respect to the current deformation gradient
$\mathbf F^{n+1}$, while treating all history quantities at $t_n$ as constants within the Newton iteration.

\paragraph{Algorithmic form of the stress update (clear separation of current vs.\ history).}
Insert the recursion \eqref{eq:Q_update} into the total stress \eqref{eq:S_total_prony}:
\begin{align}
  \mathbf S^{n+1}
  &= \mathbf S_{\mathrm{vol}}^{n+1}
   + g_{\infty}\,\mathbf S_{\mathrm{iso},0}^{n+1}
   + \sum_{m=1}^{N_p}\Big[
        \alpha_m\mathbf Q_m^{n} + \beta_m(\mathbf S_{\mathrm{iso},0}^{n+1}-\mathbf S_{\mathrm{iso},0}^{n})
     \Big] \nonumber\\
  &= \mathbf S_{\mathrm{vol}}^{n+1}
   + \gamma\,\mathbf S_{\mathrm{iso},0}^{n+1}
   + \mathbf S_{\mathrm{hist}},
  \label{eq:S_alg_split}
\end{align}
where we define the scalar
\begin{equation}
  \gamma := g_{\infty} + \sum_{m=1}^{N_p}\beta_m,
  \label{eq:gamma_def}
\end{equation}
and the \emph{history stress} (constant during the Newton iteration at $t_{n+1}$)
\begin{equation}
  \mathbf S_{\mathrm{hist}}
  := \sum_{m=1}^{N_p}\alpha_m\mathbf Q_m^{n}
   - \left(\sum_{m=1}^{N_p}\beta_m\right)\mathbf S_{\mathrm{iso},0}^{n}.
  \label{eq:S_hist_def}
\end{equation}
With \eqref{eq:P_total_prony_compact}, the total first Piola stress becomes
\begin{equation}
  \mathbf P^{n+1}
  = \mathbf F^{n+1}\Big(\mathbf S_{\mathrm{vol}}^{n+1} + \gamma\,\mathbf S_{\mathrm{iso},0}^{n+1}\Big)
  + \mathbf F^{n+1}\mathbf S_{\mathrm{hist}}
  =: \mathbf P_{\mathrm{alg}}^{n+1} + \mathbf F^{n+1}\mathbf S_{\mathrm{hist}} .
  \label{eq:P_alg_plus_hist}
\end{equation}
The term $\mathbf P_{\mathrm{alg}}^{n+1}$ collects all contributions that depend on the current iterate
$\mathbf F^{n+1}$ through $\mathbf S_{\mathrm{vol}}^{n+1}$ and $\mathbf S_{\mathrm{iso},0}^{n+1}$, whereas the
history term $\mathbf F^{n+1}\mathbf S_{\mathrm{hist}}$ contains only known tensors from $t_n$.

\paragraph{Interpretation via an algorithmic energy.}
Since $\gamma$ is a scalar constant for a given step $(\Delta t_r, T)$, the algorithmic stress
$\mathbf P_{\mathrm{alg}}^{n+1}$ can be obtained from the \emph{algorithmic stored energy}
\begin{equation}
  W_{\mathrm{alg}}(\mathbf F)
  := W_{\mathrm{vol}}(J) + \gamma\,W_{\mathrm{iso}}(\bar{\mathbf C}),
  \qquad
  \mathbf P_{\mathrm{alg}} = \frac{\partial W_{\mathrm{alg}}}{\partial \mathbf F}.
  \label{eq:W_alg_def}
\end{equation}
This observation is convenient for implementation: one can reuse the hyperelastic derivatives of
$W_{\mathrm{vol}}$ and $W_{\mathrm{iso}}$, with the isochoric part scaled by $\gamma$.

\paragraph{Consistent linearization.}
During the Newton iteration at $t_{n+1}$, all history quantities are fixed:
$\delta \mathbf S_{\mathrm{hist}} = \mathbf 0$.
Taking the variation of \eqref{eq:P_alg_plus_hist} therefore gives
\begin{equation}
  \delta \mathbf P^{n+1}
  = \delta \mathbf P_{\mathrm{alg}}^{n+1}
  + \delta\mathbf F^{n+1}\,\mathbf S_{\mathrm{hist}}.
  \label{eq:dP_simple}
\end{equation}
Introduce the fourth-order algorithmic tangent, which is material Jacobian
\begin{equation}
  \mathbb A_{\mathrm{alg}}^{n+1} := \frac{\partial \mathbf P_{\mathrm{alg}}^{n+1}}{\partial \mathbf F^{n+1}},
  \label{eq:Aalg_def}
\end{equation}
which is the standard hyperelastic tangent associated with $W_{\mathrm{alg}}$ in \eqref{eq:W_alg_def}.
Then \eqref{eq:dP_simple} can be written as
\begin{equation}
  \delta \mathbf P^{n+1}
  = \Big(\mathbb A_{\mathrm{alg}}^{n+1} + \mathbb I \otimes \mathbf S_{\mathrm{hist}}\Big) : \delta \mathbf F^{n+1},
  \label{eq:A_total_compact}
\end{equation}
where $(\mathbb I \otimes \mathbf S_{\mathrm{hist}}):\delta\mathbf F = \delta\mathbf F\,\mathbf S_{\mathrm{hist}}$.
% In components,
% \begin{equation}
%   (\mathbb I \otimes \mathbf S_{\mathrm{hist}})_{iI kL} = \delta_{ik}\,(\mathbf S_{\mathrm{hist}})_{LI}.
%   \label{eq:I_otimes_Sh}
% \end{equation}
% Equation \eqref{eq:A_total_compact} is the consistent constitutive Jacobian needed by the global Newton solve
% in Sec.~2.3.6 (Newmark-$\beta$ dynamics).


\subsection{Weak form and discrete operators (Total Lagrangian)}
\label{sec:weak_form_discrete_ops}

We work in the reference configuration $\Omega_0\subset\mathbb{R}^3$ with boundary
$\partial\Omega_0=\Gamma_{u,0}\cup\Gamma_{t,0}$.
Let $\mathbf u(\mathbf X,t)$ denote the displacement and $\mathbf F(\mathbf X,t)=\mathbf I+\nabla_{\!X}\mathbf u$
the deformation gradient. The balance of linear momentum in $\Omega_0$ reads
\begin{equation}
  \rho_0\,\ddot{\mathbf u} = \mathrm{Div}_{X}\,\mathbf P + \rho_0\,\mathbf b_0
  \qquad \text{in } \Omega_0,
  \label{eq:strong_momentum_TL}
\end{equation}
with reference density $\rho_0$ and prescribed body force $\mathbf b_0$.
The boundary conditions are
\begin{equation}
  \mathbf u=\bar{\mathbf u}\ \text{on }\Gamma_{u,0},
  \qquad
  \mathbf P\,\mathbf N = \bar{\mathbf t}_0\ \text{on }\Gamma_{t,0},
  \label{eq:bc_TL}
\end{equation}
where $\mathbf N$ is the outward unit normal on $\partial\Omega_0$.
Here $\mathbf P$ denotes the \emph{total} first Piola--Kirchhoff stress.
In a time-discrete setting, $\mathbf P^{n+1}$ is the stress returned by the constitutive update
(Secs.~2.3.2--2.3.4), i.e.\ a (numerical) approximation of the total stress at $t_{n+1}$.

\paragraph{Weak form.}
Let $\mathbf w$ be an admissible virtual displacement with $\mathbf w=\mathbf 0$ on $\Gamma_{u,0}$.
Multiplying \eqref{eq:strong_momentum_TL} by $\mathbf w$ and integrating by parts yields the Total Lagrangian weak form:
find $\mathbf u(\cdot,t)$ such that for all $\mathbf w$,
\begin{equation}
  \int_{\Omega_0} \rho_0\,\mathbf w\cdot \ddot{\mathbf u}\, \mathrm dV
  + \int_{\Omega_0} (\nabla_{\!X}\mathbf w):\mathbf P \, \mathrm dV
  =
  \int_{\Omega_0} \rho_0\,\mathbf w\cdot\mathbf b_0\, \mathrm dV
  + \int_{\Gamma_{t,0}} \mathbf w\cdot \bar{\mathbf t}_0\, \mathrm dA .
  \label{eq:weak_form_TL}
\end{equation}
The second term defines the internal virtual work and will generate the discrete internal force and tangent.

\paragraph{Finite element discretization.}
Let $\{N_a(\mathbf X)\}_{a=1}^{n_{node}}$ be nodal basis functions. We approximate
\begin{equation}
  \mathbf u_h(\mathbf X,t) = \sum_{a=1}^{n_{node}} N_a(\mathbf X)\,\mathbf d_a(t),
  \qquad
  \mathbf w_h(\mathbf X) = \sum_{a=1}^{n_{node}} N_a(\mathbf X)\,\delta\mathbf d_a,
  \label{eq:FE_ansatz}
\end{equation}
where $\mathbf d_a(t)\in\mathbb{R}^3$ are nodal unknowns.
Substituting \eqref{eq:FE_ansatz} into \eqref{eq:weak_form_TL} yields the semi-discrete system
\begin{equation}
  \mathbf M\,\ddot{\mathbf d}(t) + \mathbf f_{\mathrm{int}}(\mathbf d(t),\mathrm{history})
  =
  \mathbf f_{\mathrm{ext}}(t),
  \label{eq:semi_discrete}
\end{equation}
with global vector $\mathbf d=[\mathbf d_1;\dots;\mathbf d_{n_{node}}]\in\mathbb{R}^{3n_{node}}$.
The consistent mass matrix and external force are
\begin{align}
  \mathbf M_{ab} &= \int_{\Omega_0} \rho_0\,N_a N_b\,\mathbf I\,\mathrm dV,
  \label{eq:mass_matrix}\\
  (\mathbf f_{\mathrm{ext}})_a &= \int_{\Omega_0} \rho_0\,N_a\,\mathbf b_0\,\mathrm dV
  + \int_{\Gamma_{t,0}} N_a\,\bar{\mathbf t}_0\,\mathrm dA .
  \label{eq:fext}
\end{align}

\paragraph{Internal force in index form.}
Let $\mathbf P=\mathbf P(\mathbf F;\mathrm{history})$ be evaluated pointwise in $\Omega_0$ by the constitutive update.
From \eqref{eq:weak_form_TL}, the internal nodal force is
\begin{equation}
  (\mathbf f_{\mathrm{int}})_a
  = \int_{\Omega_0} (\nabla_{\!X} N_a)\cdot \mathbf P \, \mathrm dV,
  \qquad
  \text{i.e.}\quad
  (f_{\mathrm{int}})_a^{\,i} = \int_{\Omega_0} \frac{\partial N_a}{\partial X_I}\,P_{iI}\,\mathrm dV,
  \label{eq:fint_index}
\end{equation}
where $i$ is a spatial component index and $I$ a reference-coordinate index.

% ============================================================
% REPLACE ONLY THE SCREENSHOT-COVERED PART OF SEC. 2.3.5 WITH THIS
% ============================================================


\paragraph{Internal force (Total Lagrangian).}
With the Total Lagrangian weak form, the internal nodal force associated with node $a$ is
\begin{equation}
  (\mathbf f_{\mathrm{int}})_a(\mathbf d,\mathrm{history})
  = \int_{\Omega_0} \mathbf P(\mathbf F(\mathbf d),\mathrm{history})\,\nabla_{\!X}N_a \, \mathrm dV,
  \label{eq:fint_def}
\end{equation}
where $\nabla_{\!X}$ denotes the gradient with respect to the reference coordinate $\mathbf X$.
In components,
\begin{equation}
  (f_{\mathrm{int}})_a^{\,m}
  = \int_{\Omega_0} P_{mI}\,\frac{\partial N_a}{\partial X_I}\,\mathrm dV,
  \qquad m=1,2,3.
  \label{eq:fint_comp}
\end{equation}

\paragraph{Consistent tangent and element stiffness.}
The material contribution to the element stiffness follows from linearizing \eqref{eq:fint_comp}
with respect to the nodal degrees of freedom. Define the element block
\begin{equation}
  K_{ab}^{\,mk} := \frac{\partial (f_{\mathrm{int}})_a^{\,m}}{\partial d_b^{\,k}}.
  \label{eq:K_def}
\end{equation}
Differentiating \eqref{eq:fint_comp} gives
\begin{equation}
  K_{ab}^{\,mk}
  = \int_{\Omega_0}
  \frac{\partial P_{mI}}{\partial d_b^{\,k}}
  \frac{\partial N_a}{\partial X_I}\,\mathrm dV.
  \label{eq:K_step1}
\end{equation}
Using the chain rule through the deformation gradient,
\begin{equation}
  \frac{\partial P_{mI}}{\partial d_b^{\,k}}
  = \sum_{p,Q}\frac{\partial P_{mI}}{\partial F_{pQ}}
  \frac{\partial F_{pQ}}{\partial d_b^{\,k}}
  = \sum_{p,Q} A_{mI pQ}\,\frac{\partial F_{pQ}}{\partial d_b^{\,k}},
  \qquad
  A_{mI pQ} := \frac{\partial P_{mI}}{\partial F_{pQ}},
  \label{eq:chain_dP_dd}
\end{equation}
where $\mathbb A=\partial\mathbf P/\partial\mathbf F$ is the (algorithmic) Total Lagrangian tangent
used by Newton's method.

\noindent Next, express $\mathbf F$ in terms of nodal unknowns. Since
$\mathbf x(\mathbf X)=\mathbf X+\mathbf u_h(\mathbf X)$, one has
\begin{equation}
  F_{pQ} = \frac{\partial x_p}{\partial X_Q}
  = \delta_{pQ} + \sum_{c=1}^{n_{\mathrm{node}}} d_c^{\,p}\,\frac{\partial N_c}{\partial X_Q},
  \qquad
  \frac{\partial F_{pQ}}{\partial d_b^{\,k}}
  = \delta_{pk}\,\frac{\partial N_b}{\partial X_Q}.
  \label{eq:dF_dd}
\end{equation}
Substitution of \eqref{eq:dF_dd} into \eqref{eq:chain_dP_dd} yields
\begin{equation}
  \frac{\partial P_{mI}}{\partial d_b^{\,k}}
  = \sum_{Q} A_{mI kQ}\,\frac{\partial N_b}{\partial X_Q}.
  \label{eq:dP_dd_final}
\end{equation}
Finally, inserting \eqref{eq:dP_dd_final} into \eqref{eq:K_step1} gives the standard Total Lagrangian
stiffness contraction
\begin{equation}
  K_{ab}^{\,mk}
  = \int_{\Omega_0}
  \sum_{I,Q}
  \frac{\partial N_a}{\partial X_I}\,
  A_{mI kQ}\,
  \frac{\partial N_b}{\partial X_Q}\,
  \mathrm dV.
  \label{eq:K_TL_final}
\end{equation}

\paragraph{Split of the 4th-order tangent $\mathbb A$.}
Within the Newton iterations of a fixed time step $t_n\to t_{n+1}$, all state variables stored at $t_n$
are constants. Using the Prony recursion in Sec.~2.3.3, the second Piola stress at $t_{n+1}$ can be written as
\begin{equation}
  \mathbf S^{n+1} = \mathbf S_{\mathrm{alg}}^{n+1} + \mathbf S_{\mathrm{hist}},
  \label{eq:S_alg_hist_split}
\end{equation}
where $\mathbf S_{\mathrm{hist}}$ collects the history-only terms that are constant during the Newton iterations.
For the update
$\mathbf Q_m^{n+1}=\alpha_m \mathbf Q_m^n+\beta_m(\mathbf S_{\mathrm{iso},0}^{n+1}-\mathbf S_{\mathrm{iso},0}^{n})$,
the decomposition \eqref{eq:S_alg_hist_split} is obtained by defining
\begin{equation}
  \mathbf S_{\mathrm{alg}}^{n+1}
  := \mathbf S_{\mathrm{vol}}^{n+1} + \gamma\,\mathbf S_{\mathrm{iso},0}^{n+1},
  \qquad
  \mathbf S_{\mathrm{hist}}
  := \sum_{m=1}^{N_p}\alpha_m \mathbf Q_m^n - \Big(\sum_{m=1}^{N_p}\beta_m\Big)\mathbf S_{\mathrm{iso},0}^{n},
  \qquad
  \gamma := g_\infty + \sum_{m=1}^{N_p}\beta_m,
  \label{eq:S_alg_S_hist_def}
\end{equation}
so that $\mathbf S^{n+1}=\mathbf S_{\mathrm{vol}}^{n+1}+g_\infty\mathbf S_{\mathrm{iso},0}^{n+1}+\sum_m \mathbf Q_m^{n+1}$ holds identically.

\noindent Since $\mathbf P^{n+1}=\mathbf F^{n+1}\mathbf S^{n+1}$,
\begin{equation}
  \mathbf P^{n+1} = \mathbf F^{n+1}\mathbf S_{\mathrm{alg}}^{n+1} + \mathbf F^{n+1}\mathbf S_{\mathrm{hist}}
  =: \mathbf P_{\mathrm{alg}}^{n+1} + \mathbf F^{n+1}\mathbf S_{\mathrm{hist}}.
  \label{eq:P_alg_hist_split}
\end{equation}
Taking variations yields
$\delta\mathbf P^{n+1}=\delta\mathbf P_{\mathrm{alg}}^{n+1}+\delta\mathbf F^{n+1}\mathbf S_{\mathrm{hist}}$,
hence the tangent decomposes as
\begin{equation}
  \mathbb A
  := \frac{\partial\mathbf P^{n+1}}{\partial\mathbf F^{n+1}}
  = \mathbb A_{\mathrm{alg}} + \mathbb I\otimes \mathbf S_{\mathrm{hist}},
  \qquad
  (\mathbb I\otimes \mathbf S_{\mathrm{hist}})_{mI kQ}=\delta_{mk}(\mathbf S_{\mathrm{hist}})_{QI}.
  \label{eq:A_split}
\end{equation}

\paragraph{Explicit form of $\mathbb A_{\mathrm{alg}}$.}
From $\mathbf P_{\mathrm{alg}}=\mathbf F\,\mathbf S_{\mathrm{alg}}$,
\begin{equation}
  \delta\mathbf P_{\mathrm{alg}}
  = \delta\mathbf F\,\mathbf S_{\mathrm{alg}} + \mathbf F\,\delta\mathbf S_{\mathrm{alg}}.
  \label{eq:dPalg}
\end{equation}
If the constitutive routine provides the second-Piola tangent
$\delta\mathbf S_{\mathrm{alg}}=\mathbb C_{\mathrm{alg}}:\delta\mathbf E$ with
$\mathbf E=\tfrac12(\mathbf C-\mathbf I)$ and $\delta\mathbf E=\tfrac12(\mathbf F^T\delta\mathbf F+(\delta\mathbf F)^T\mathbf F)$,
then $\delta\mathbf P_{\mathrm{alg}}=\mathbb A_{\mathrm{alg}}:\delta\mathbf F$ holds with
\begin{equation}
  (\mathbb A_{\mathrm{alg}})_{mI kQ}
  = \delta_{mk}(\mathbf S_{\mathrm{alg}})_{QI}
  + \frac12\,F_{mJ}\,(\mathbb C_{\mathrm{alg}})_{JI MN}\,
  \big(\delta_{Q N}F_{kM} + \delta_{QM}F_{kN}\big),
  \label{eq:Aalg_comp}
\end{equation}
evaluated at the current Newton iterate. Combining \eqref{eq:A_split} and \eqref{eq:Aalg_comp} gives the full
$\mathbb A$ used in \eqref{eq:K_TL_final}.

\paragraph{Reference-element mapping and precomputed transformed tangent.}
On a reference element with coordinates $\boldsymbol\xi$, let
$\mathbf J:=\partial\mathbf X/\partial\boldsymbol\xi$ be the geometric mapping Jacobian. Then
\begin{equation}
  \frac{\partial N_a}{\partial X_I}
  = \sum_{r}\frac{\partial N_a}{\partial \xi_r}\,(\mathbf J^{-1})_{rI},
  \qquad
  \mathrm dV=\det(\mathbf J)\,\mathrm dV_{\xi}.
  \label{eq:ref_map}
\end{equation}
Substituting \eqref{eq:ref_map} into \eqref{eq:K_TL_final} motivates the transformed tensor
\begin{equation}
  \widetilde A_{m k r q}
  := \sum_{I,Q} A_{mI kQ}\,(\mathbf J^{-1})_{rI}\,(\mathbf J^{-1})_{qQ}\,\det(\mathbf J),
  \label{eq:A_tilde}
\end{equation}
so that the element stiffness can be written as
\begin{equation}
  K_{ab}^{\,mk}
  = \int_{\Omega_{\xi}}
  \sum_{r,q}\widetilde A_{m k r q}\,
  \frac{\partial N_a}{\partial \xi_r}\,
  \frac{\partial N_b}{\partial \xi_q}\,
  \mathrm dV_{\xi}.
  \label{eq:K_ref}
\end{equation}


\subsection{Time integration: Newmark scheme and Newton--Raphson iteration}
\label{sec:newmark_newton}

We advance the semi-discrete hyper--viscoelastodynamic system from Sec.~2.3.5,
\begin{equation}
  \mathbf M\,\ddot{\mathbf d}(t) + \mathbf f_{\mathrm{int}}\big(\mathbf d(t),\mathrm{history}(t)\big)
  = \mathbf f_{\mathrm{ext}}(t),
  \label{eq:semi_discrete_dyn}
\end{equation}
where $\mathbf d(t)\in\mathbb R^{n_{\mathrm{dof}}}$ collects the nodal unknowns.
The internal force $\mathbf f_{\mathrm{int}}$ is assembled from the \emph{total} first
Piola--Kirchhoff stress $\mathbf P$ delivered by the constitutive update (Secs.~2.3.2--2.3.4),
together with the consistent Total Lagrangian tangent $\mathbb A:=\partial\mathbf P/\partial\mathbf F$
used for stiffness assembly in Sec.~2.3.5.

\paragraph{Time grid and Newmark kinematics.}
Let $t_{n+1}=t_n+\Delta t$ and denote the converged state at $t_n$ by
$\{\mathbf d^n,\mathbf v^n,\mathbf a^n\}$ (displacement, velocity, acceleration).
We employ the Newmark updating framework with parameters $(\beta_N,\gamma_N)$:
\begin{align}
  \mathbf d^{n+1} &=
  \mathbf d^n + \Delta t\,\mathbf v^n
  + \Delta t^2\Big(\tfrac{1}{2}-\beta_N\Big)\mathbf a^n
  + \Delta t^2\,\beta_N\,\mathbf a^{n+1},
  \label{eq:newmark_d}\\
  \mathbf v^{n+1} &=
  \mathbf v^n + \Delta t\Big((1-\gamma_N)\mathbf a^n + \gamma_N\,\mathbf a^{n+1}\Big).
  \label{eq:newmark_v}
\end{align}
In the implicit setting, $\mathbf d^{n+1}$ is taken as the primary unknown and
$\mathbf a^{n+1}$, $\mathbf v^{n+1}$ are expressed as functions of $\mathbf d^{n+1}$:
\begin{align}
  \mathbf a^{n+1}(\mathbf d^{n+1})
  &= \frac{1}{\beta_N\,\Delta t^2}\Big(\mathbf d^{n+1}-\mathbf d^n-\Delta t\,\mathbf v^n\Big)
   - \frac{1-2\beta_N}{2\beta_N}\,\mathbf a^n,
  \label{eq:newmark_a_of_d}\\
  \mathbf v^{n+1}(\mathbf d^{n+1})
  &= \frac{\gamma_N}{\beta_N\,\Delta t}\Big(\mathbf d^{n+1}-\mathbf d^n\Big)
   + \Big(1-\frac{\gamma_N}{\beta_N}\Big)\mathbf v^n
   + \Delta t\Big(1-\frac{\gamma_N}{2\beta_N}\Big)\mathbf a^n.
  \label{eq:newmark_v_of_d}
\end{align}
A common choice is $(\beta_N,\gamma_N)=(1/4,1/2)$, which is the average acceleration.

\paragraph{Nonlinear residual at $t_{n+1}$.}
Evaluating \eqref{eq:semi_discrete_dyn} at $t_{n+1}$ and substituting
\eqref{eq:newmark_a_of_d} gives the nonlinear system for $\mathbf d^{n+1}$:
\begin{equation}
  \mathbf R^{n+1}(\mathbf d^{n+1})
  := \mathbf M\,\mathbf a^{n+1}(\mathbf d^{n+1})
   + \mathbf f_{\mathrm{int}}\!\big(\mathbf d^{n+1},\mathrm{history}_n\big)
   - \mathbf f_{\mathrm{ext}}^{n+1}
  = \mathbf 0.
  \label{eq:residual_newmark}
\end{equation}
Here $\mathrm{history}_n$ denotes the stored internal variables at $t_n$ (e.g.\ Prony overstresses
$\{\mathbf Q_m^n\}$). Within the Newton iterations of the step $t_n\to t_{n+1}$,
$\mathrm{history}_n$ is fixed, while the constitutive update provides
$\mathbf P^{n+1}$ and $\mathbb A^{n+1}$ consistently with the time-discrete internal-variable recursion.

\paragraph{Newton--Raphson linearization.}
Given an iterate $\mathbf d^{n+1,(k)}$, we solve for the correction $\Delta\mathbf d^{(k)}$ via
\begin{equation}
  \mathbf J^{(k)}\,\Delta\mathbf d^{(k)} = -\mathbf R^{(k)},
  \qquad
  \mathbf d^{n+1,(k+1)} = \mathbf d^{n+1,(k)} + \Delta\mathbf d^{(k)},
  \label{eq:newton_update}
\end{equation}
where $\mathbf R^{(k)}:=\mathbf R^{n+1}(\mathbf d^{n+1,(k)})$ and
$\mathbf J^{(k)}:=\partial\mathbf R^{n+1}/\partial\mathbf d^{n+1}$.

\noindent Using \eqref{eq:newmark_a_of_d}, the variation of the acceleration is
\begin{equation}
  \delta\mathbf a^{n+1} = \frac{1}{\beta_N\,\Delta t^2}\,\delta\mathbf d^{n+1}.
  \label{eq:da_dd}
\end{equation}
Moreover, the consistent internal-force linearization defines
\begin{equation}
  \delta \mathbf f_{\mathrm{int}}
  = \mathbf K_{\mathrm{int}}\,\delta\mathbf d^{n+1},
  \qquad
  \mathbf K_{\mathrm{int}} := \frac{\partial \mathbf f_{\mathrm{int}}}{\partial \mathbf d^{n+1}},
  \label{eq:Kint_def}
\end{equation}
with $\mathbf K_{\mathrm{int}}$ assembled element-wise from $\mathbb A=\partial\mathbf P/\partial\mathbf F$
as detailed in Sec.~2.3.5.
Combining \eqref{eq:da_dd}--\eqref{eq:Kint_def} yields the Newmark--Newton Jacobian
\begin{equation}
  \mathbf J^{(k)}
  = \frac{1}{\beta_N\,\Delta t^2}\,\mathbf M + \mathbf K_{\mathrm{int}}^{(k)}.
  \label{eq:newmark_jacobian}
\end{equation}
This is the effective tangent used in \eqref{eq:newton_update}.




\paragraph{Algorithm for Newton-Raphson Iteration.}
Given $(\mathbf d^n,\mathbf v^n,\mathbf a^n,\mathrm{history}_n)$:
\begin{enumerate}
  \item \textbf{Predictor (optional).} Set an initial guess, e.g.\ $\mathbf d^{n+1,(0)}=\mathbf d^n+\Delta t\,\mathbf v^n$.
  \item \textbf{Newton loop} for $k=0,1,\dots$:
  \begin{enumerate}
    \item Compute $\mathbf a^{n+1,(k)}$ and $\mathbf v^{n+1,(k)}$ from \eqref{eq:newmark_a_of_d}--\eqref{eq:newmark_v_of_d}.
    \item At each quadrature point, build $\mathbf F^{n+1,(k)}$ from $\mathbf d^{n+1,(k)}$ (Sec.~2.3.1),
          call the constitutive update (Secs.~2.3.3--2.3.4) to obtain
          $\mathbf P^{n+1,(k)}$ and $\mathbb A^{n+1,(k)}$.
    \item Assemble $\mathbf f_{\mathrm{int}}^{(k)}$ and $\mathbf K_{\mathrm{int}}^{(k)}$ (Sec.~2.3.5),
          then residual $\mathbf R^{(k)}$ from \eqref{eq:residual_newmark}.
    \item Assemble Jacobian $\mathbf J^{(k)}$ from \eqref{eq:newmark_jacobian} and solve
          $\mathbf J^{(k)}\Delta\mathbf d^{(k)}=-\mathbf R^{(k)}$.
    \item Update $\mathbf d^{n+1,(k+1)}=\mathbf d^{n+1,(k)}+\Delta\mathbf d^{(k)}$.
  \end{enumerate}
  \item \textbf{Accept step.} Upon convergence, set
  $\mathbf d^{n+1}=\mathbf d^{n+1,(k_\star)}$ and compute
  $\mathbf a^{n+1},\mathbf v^{n+1}$ from \eqref{eq:newmark_a_of_d}--\eqref{eq:newmark_v_of_d}.
  Store the internal variables at $t_{n+1}$ as $\mathrm{history}_{n+1}$.
\end{enumerate}






\chapter{Implementation and Code Organization}
\label{ch:implementation}

\section{Implementation overview and execution flow}
\label{sec:implementation_overview}

This chapter describes how the continuum model and time-discrete algorithms introduced in Chapter~2 are mapped to a practical, high-performance implementation in the SFEM code base.  The presentation emphasizes responsibilities and I/O at each software layer, the timing of history accesses during Newton--Newmark solves, and the minimal set of primitives that a constitutive routine must provide to integrate smoothly with element kernels and global assembly.

\noindent This implementation is built on top of the SFEM codebase. The SFEM library provides the project infrastructure (mesh, element loops, I/O and solver interfaces) that this work reuses and extends; in particular, core finite‑element data structures and element dispatching are taken from SFEM and augmented here with the finite‑strain hyper‑viscoelastic constitutive updates and the GPU/offload kernels described below. For reproducibility the experiments in this thesis are based on the SFEM repository (see \cite{sfemgit}) and the specific modifications are documented in the implementation notes and accompanying repository. A formal citation for the SFEM codebase is included in the bibliography.

\noindent The implementation follows a layered execution flow. A single implicit time step from $t_n\to t_{n+1}$ can be viewed as the sequence:

\begin{center}
  time step $\to$ Newmark predictor $\to$ Newton loop $\to$ element kernels $\to$ global assemble/solve $\to$ state commit
\end{center}

\noindent Key points about data movement and timing:

\begin{itemize}
  \item \textbf{History read/write semantics.} All history  variables are stored at quadrature points on the reference configuration (see Sec.~\ref{sec:fs_visco_prony}).  Within a Newton iteration for the step $t_n\to t_{n+1}$ the history values corresponding to $t_n$ (e.g. $\{\mathbf Q_m^n\}$) are to be treated as constants: they may be read by the constitutive update but must not be overwritten. Only after the Newton step is accepted are the updated history values at $t_{n+1}$ written back. This preserves the Total Lagrangian consistency of the algorithm and avoids accidental interleaving of partially updated state across iterations.

  \item \textbf{Implementation mode: FLEXIBLE (unique\_Hi).} The SFEM implementation used in this thesis employs the FLEXIBLE history-storage strategy: only the active Prony branch histories $H_i^n$ are stored at each quadrature point and the contribution of fully relaxed branches is absorbed into a scalar parameter $\gamma$. Kernels produced by the code generator are therefore written in a compact, parameterized form (algorithmic part depends on $\gamma$) while history-related contributions and updates are performed by small runtime loops over active terms. This design reduces per‑qp memory, simplifies support for variable numbers of Prony terms, and keeps generated C kernels concise.

  \item \textbf{History layout and buffers.} The frontend (\texttt{\detokenize{MooneyRivlinVisco}} operator) allocates host-side history buffers sized by the number of elements, quadrature points, and the per‑qp history stride required by the FLEXIBLE layout. A \texttt{new\_history} buffer is written during the step and swapped into place atomically via \texttt{swap\_history\_buffers()} on step acceptance. An auxiliary \texttt{prev\_u} buffer stores the previous displacement field when needed by history updates.

  \item \textbf{Total Lagrangian convention.} All constitutive computations and history variables live in the reference configuration $\Omega_0$ (Sec.~\ref{sec:finite_strain_kinematics}). The constitutive update therefore accepts the current deformation gradient $\mathbf F^{n+1,(k)}$, returns the pointwise first Piola stress $\mathbf P^{n+1,(k)}$ and (optionally) the algorithmic tangent $\mathbb A^{n+1,(k)}$, and keeps any temporary work memory local to the kernel.

  \item \textbf{Separation of algorithmic and history contributions.} As derived in Sec.~2.3.3--2.3.4, the constitutive model produces an \emph{algorithmic} stress contribution (that depends on the current iterate) and a history-only contribution that is constant during Newton iterations. Implementations should exploit this split to avoid recomputing history-dependent scalars inside tight loops and to form the material tangent as
  \[ \mathbb A = \mathbb A_{\mathrm{alg}} + \mathbb I \otimes \mathbf S_{\mathrm{hist}}, \]
  consistent with Eq.~\eqref{eq:A_split}.

  \item \textbf{Code-generation and kernel decomposition.} The project uses a symbolic code-generation pipeline (\texttt{sr\_visco\_hyper\_unique\_Hi.py}) that emits per‑qp C kernels for the algorithmic parts and small helper kernels for history handling. Typical generated kernels (for \texttt{hex8} elements and the flexible architecture) include:
  \begin{itemize}
    \item \texttt{\detokenize{hex8_mooney_rivlin_S_dev_from_disp}}: compute elastic deviatoric stress $\mathbf S_{\mathrm{dev}}$ from the displacement (no history access);
    \item \texttt{\detokenize{hex8_mooney_rivlin_grad_flexible}}: algorithmic gradient with symbolic $\gamma$ (no history writes);
    \item \texttt{\detokenize{hex8_mooney_rivlin_grad_hist_single}}: single‑Prony‑term history contribution to gradient (called in a loop over active terms at runtime);
    \item \texttt{\detokenize{hex8_mooney_rivlin_hessian_algo_micro}} / \texttt{\detokenize{hex8_mooney_rivlin_S_lin_flexible}}: algorithmic Hessian / metric tensor generation (history geometric stiffness added at runtime).
  \end{itemize}
  This decomposition keeps generated code compact and lets the C driver loop over Prony terms to add history contributions and to update history entries.

  \item \textbf{Prony precomputation and parameter passing.} The frontend computes WLF-shifted effective relaxation times and classifies Prony terms into ``active'' and ``fully relaxed'' when the operator is configured. In typical use, which is fixed temperature and unchanged Prony data, this computation happens once at setup; it is recomputed only if the time step, Prony series, WLF parameters or temperature are modified. Active terms yield arrays of $\alpha$ and $\beta$ passed to the kernels; fully relaxed weights are absorbed into \texttt{\detokenize{gamma}}. Generated kernels therefore accept \texttt{\detokenize{alpha}},\texttt{\detokenize{beta}} arrays and \texttt{\detokenize{gamma}}.


  \item \textbf{Minimal constitutive routine contract.} A well-defined constitutive routine (used by element kernels) should have the following interface semantics:
  \begin{enumerate}
    \item Inputs: quadrature point geometry, current deformation gradient $\mathbf F^{n+1,(k)}$, time step parameters $(\Delta t, a_T)$, and the history bundle $\mathrm{history}_n$.
    \item Outputs: pointwise first Piola stress $\mathbf P^{n+1,(k)}$, the algorithmic material tangent (optionally) $\mathbb A_{\mathrm{alg}}^{n+1,(k)}$, and any per-qp diagnostics for testing (energies, $J$, $\mathbf S$, etc.).
    \item Side-effects: none during Newton iterations. A separate ``commit'' call must be provided that atomically replaces $\mathrm{history}_n$ by $\mathrm{history}_{n+1}$ once the step is accepted.
  \end{enumerate}

  \item \textbf{High-level driver responsibilities.} The top-level driver (time loop) orchestrates Newmark prediction, Newton iterations and the accept/commit logic. The driver:
  \begin{itemize}
    \item forms the Newmark predictor for $\mathbf d^{n+1,(0)}$ and computes the associated $\mathbf a^{n+1,(0)}$, $\mathbf v^{n+1,(0)}$ (Eqs.~\eqref{eq:newmark_d}--\eqref{eq:newmark_v});
    \item enters the Newton loop and for each iterate calls element kernels to evaluate $\mathbf f_{\mathrm{int}}$ and $\mathbf K_{\mathrm{int}}$;
    \item assembles the global residual $\mathbf R^{(k)}$ and Jacobian $\mathbf J^{(k)}$ (Eq.~\eqref{eq:newmark_jacobian}) and invokes the linear solver; and
    \item upon convergence invokes the commit operation to persist updated history variables.
  \end{itemize}

  \item \textbf{Element kernel responsibilities.} An element kernel performs operations at the element level and across quadrature points: it
  \begin{enumerate}
    \item evaluates shape function gradients and builds $\mathbf F^{n+1,(k)}$ at each qp (using nodal dofs);
    \item calls the constitutive update to obtain $\mathbf P^{n+1,(k)}$ and $\mathbb A^{n+1,(k)}$ (or the parts required to form the transformed tangent $\widetilde A$ in Eq.~\eqref{eq:A_tilde});
    \item forms element contributions to the internal force and stiffness (Eqs.~\eqref{eq:fint_def},~\eqref{eq:K_ref}); and
    \item exposes qp-level diagnostics for unit tests and debugging (energies, $\mathbf S$, $\mathbf P$, $J$, etc.).
  \end{enumerate}

  \item \textbf{I/O and debugging hooks.} For verification and regression testing the implementation should provide optional per-qp or per-element dumps of: the deformation measures $(\mathbf F, J, \bar{\mathbf C})$, the returned stresses $(\mathbf S,\mathbf P)$, the material tangent components, and energy increments. These hooks are invaluable when comparing analytic tangents against finite-difference checks (see Sec.~\ref{sec:alg_stress_consistent_lin}).
\end{itemize}

\paragraph{Why this structure matters.} The layered separation (driver $\to$ element kernels $\to$ constitutive routine) keeps responsibilities narrow, simplifies unit testing, and enables multiple backend strategies: the same constitutive routine can be used in a CPU element kernel, a vectorized batch kernel, or a GPU kernel provided the history-access semantics are respected. Moreover, keeping the algorithmic stress derivation in the constitutive update and exposing a small, stable interface reduces duplication and the risk of inconsistencies between energy and tangent.

\bigskip

\noindent The remainder of this chapter gives a "top-down" description of the code structure: Section~3.2 presents the call chain and module responsibilities; Section~3.3 describes the code-generation pipeline used for pointwise kernels; Section~3.4 discusses data layouts and history storage; Section~3.5 describes kernel design and CPU/GPU split; Section~3.6 maps index notation to concrete code arrays; Section~3.7 details the Newton--Newmark driver in code; and Sections~3.8--3.9 cover testing hooks and performance notes.























\appendix %optional, use only if you have an appendix

\chapter{Some retarded material}

\backmatter

\chapter{Glossary} %optional

%\bibliographystyle{alpha}
%\bibliographystyle{dcu}
\bibliographystyle{plainnat}
\bibliography{biblio}

%\cleardoublepage
%\theindex %optional, use only if you have an index, must use
	  %\makeindex in the preamble

\end{document}
